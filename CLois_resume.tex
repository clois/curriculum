\documentclass[letterpaper]{article}

\usepackage{natbib}
\usepackage{eurosym}
\usepackage{hyperref}
\usepackage{fancyhdr}
\usepackage{lastpage}
\hypersetup{pdftitle={CristinaLoisResume},
  colorlinks=true,
  linkcolor=blue,
  citecolor=blue,
  urlcolor=blue,
  pdfsubject={Cristina Lois Resume},
  pdfauthor={Cristina Lois},
  pdfpagemode={None},
  pdfstartview=FitH
}

\setlength{\parskip}{1ex}
\setlength{\parindent}{0cm}
\setlength{\oddsidemargin}{.2cm}
\setlength{\evensidemargin}{.2cm}
\setlength{\topmargin}{-2.0cm}
\setlength{\textwidth}{15cm}
\setlength{\textheight}{22cm}


\setlength{\unitlength}{1cm}

% Horizontal line
\def\hlinha#1{
	\\[-1ex]
	\begin{picture}(#1,0)
%	\thicklines
	\put(0,0){\line(1,0){#1}}
	\end{picture}
}

% Big horizontal line
\def\blinha{\hlinha{17.8}}

% Algunhas definicions
\def\bloque#1{\vspace{.0cm}\begin{large} \textbf{#1}\end{large} \blinha}
%\def\bloque#1{\vspace{.5cm}\begin{large} \textbf{#1}\end{large} \blinha}

\def\nome#1{\begin{center} \begin{Large}\textbf{#1}\end{Large}\end{center}}

\usepackage[left=0.75in,top=0.6in,right=0.75in,bottom=0.6in]{geometry} % Document margins

\pagestyle{fancy}
\fancyhf{}
\rfoot{Cristina Lois}
\renewcommand\headrulewidth{0pt}
\cfoot{\thepage~of~\pageref{LastPage}}

\begin{document}

\nome{\textsc{Cristina Lois}}

\begin{center}
16.5 Suffolk Street, Cambridge, MA 02139 (USA) \enskip \textbullet \enskip
%Massachusetts Institute of Technology\\
%Research Laboratory of Electronics, 
%Room 36-705 \\
%77 Massachusetts Ave.\\
%Cambridge, MA 02139-4307  (USA)\\
+1(617)-319-3615 \enskip \textbullet \enskip \href{mailto:clois@mit.edu}{\texttt{clois@mit.edu}}
\end{center}

\bloque{Summary}
9 years of post-doctoral experience in Biomedical Imaging, covering a wide range of topics from detector development to clinical applications. Broad expertise in PET/CT, PET/MRI, and SPECT, with clinical exposure and technical expertise in image processing and quantification.
Proven track record of successful management, innovation, and impact in bioengineering. Proven ability to work in a team environment on multi-disciplinary, international projects, as well as independently. 

\vspace{.5cm}
\bloque{Professional Experience}
\begin{description}

\item[] \textbf{Harvard Medical School \& Massachusetts General Hospital } \hfill Boston, MA\\
Research Fellow in Radiology \hfill \textit{Jul. 2015--present} 
\vspace*{-0.2cm}
\begin{itemize}
%\vspace*{-3.0ex}
 \item Responsible of PET/MR imaging analysis in a clinical study to search for markers of disease and progression in Huntington's Disease. %This is being used in a clinical trial to test new treatments.
 \item Using PET/MR neuroimaging to explore and develop new concepts in healthy and diseased brain function.
 \item Collaborating with a multi-disciplinary team of radiochemists, neuroscientists, and clinical physicians.
  
\end{itemize}

\item[] \textbf{Massachusetts Institute of Technology} \hfill Cambridge, MA \\
``M+Vision'' Research Fellow in Translational Biomedical Imaging \hfill \textit{Jul. 2012--June 2015} 
\vspace*{-0.2cm}
\begin{itemize} 
\item Intensive training on identifying unmet clinical needs and designing
    solutions with high translational potential and marketability.
 \item Conducted interviews with key opinion leaders in oncology, neurology, psychiatry, and radiology.
 \item Responsible of PET/MR imaging in a clinical study to investigate the biological basis of the placebo effect in depression.
\item Built and co-managed an international team of radiochemists, biologists, engineers, and physicians, working on the development of a new PET tracer for early assessment of treatment response in melanoma patients. 
\item Received \$283K in internal grant funding. 
\item Co-organized and taught a MIT IAP course on Biomedical Imaging.
\end{itemize}

\item[]
    \textbf{Hospital Clinic Barcelona} \hfill Barcelona, Spain 
\\
Postdoctoral Research Associate \hfill  \textit{Jan. 2012-- Jun. 2012}
\vspace*{-0.2cm}
\begin{itemize}
\item Evaluated the value of dual-time-point PET for prediction of treatment response in lung
cancer patients.
\end{itemize}

\item[] \textbf{Imaging Science Institute of T\"ubingen} \hfill T\"ubingen, Germany \\
Visiting Scientist \hfill  \textit{Feb. 2011-- Jul. 2011} 
\vspace*{-0.2cm}
\begin{itemize}
\item Analyzed the effect of MRI contrast agents on PET quantification for PET/MRI
applications and demonstrated how to avoid potential artifacts that could impact clinical decisions.
\item Evaluated artifacts caused by dental implants in PET/CT and PET/MR imaging.
\item Worked in a highly collaborative multi-disciplinary team of physicians, MR and PET physicists, and computer engineers.  
\end{itemize}
%\item[]
%    \textbf{Hospital Clinic Barcelona} \hfill Barcelona, Spain 
%\\
%Visiting Researcher \hfill  \textit{3 months in 2009 and 2010} 

%Compared the performance of GATE \& PeneloPET in simulating a microPET
%scanner.

\item[] \textbf{University of Santiago de Compostela} \hfill Santiago de Compostela, Spain\\
    ``\'Angeles Alvari\~no'' Fellow \hfill \textit{Dec. 2008--Dec 2011}
    \vspace*{-0.2cm}
   \begin{itemize} 
   \item Designed and built an affordable preclinical SPECT system by reusing a clinical gamma-camera. 
    \item Received 60K \euro~in competitive, public grant funding as PI. 
    \item Led a multi-disciplinary team of physicists, nuclear medicine physicians, and mechanical engineers.
    \item Trained and supervised a graduate student and a postdoc. 
    \item Responsible of designing and teaching one undergraduate laboratory course and two graduate courses.  
   \end{itemize} 
 %  \newpage

\item[] \textbf{University of Tennessee Medical Center} \hfill Knoxville, TN \\
Postdoctoral Research Associate \hfill \textit{Feb. 2007-- Jul. 2008} 
 \vspace*{-0.2cm}
 \begin{itemize} 
 \item Working in collaboration with Siemens, demonstrated the benefits of incorporating time-of-flight information in PET/CT by carrying out a study on a large population of 100 oncology patients. 
\item Published two highly cited papers, one chosen as cover in \textit{Journal of Nuclear Medicine}.
 \end{itemize} 
 
\item[] \textbf{University of Santiago de Compostela} \hfill Santiago de Compostela, Spain\\
    Postdoctoral Research Assistant \hfill \textit{Sep. 2006--Jan 2007}
\vspace*{-0.2cm}
\begin{itemize}
  \item  Measured the neutron fluency in a linear accelerator to estimate its
    contribution to the radiation dose in radiotherapy patients.
 \end{itemize} 

\item[] \textbf{University of Z\"urich} \hfill Z\"urich, Switzerland \\
 \&
\item[] \vspace*{-2.5ex} \textbf{University of Santiago de Compostela} \hfill Santiago de Compostela, Spain\\
    Research \& Teaching Assistant \hfill \textit{Sep. 2001--Jun. 2006}
\vspace*{-0.2cm}
\begin{itemize}
\item   Contributed to the design and development and carried out performance studies of the silicon microstrip detectors installed in the Silicon Tracker of the LHCb experiment at CERN.
\item Trained and supervised undergraduate students. 
\end{itemize} 
\end{description}

\vspace*{2.5ex}
\bloque{Education}
\begin{description}
    \item[] \textbf{PhD in Physics}, with European Doctorate mention \hfill \textit{Oct. 2001 - May 2006}\\
    University of Santiago de Compostela \hfill Santiago de Compostela, Spain 
    
    \item[] \textbf{MSc in Particle Physics \& Non-linear Dynamics} \hfill \textit{Oct. 2001 - Sep. 2003}\\
    University of Santiago de Compostela \hfill Santiago de Compostela, Spain 
    
    \item[] \textbf{BSc in Physics} \hfill \textit{Oct. 1995 - Sept. 2001} \\
    University of Santiago de Compostela \hfill Santiago de Compostela, Spain
   \end{description} 

\vspace*{2.5ex}
\bloque{Leadership service}
\begin{description}

\item Chair of the ``PET imaging'' session at the 2011 IEEE Nuclear Science Symposium and Medical Imaging Conference, Valencia, Spain.

\item Referee for  {\em Medical Physics}, {\em Physics in Medicine and Biology}, {\em Zeitschrift f\"ur Medizinische Physik}, and {\em European Journal of Nuclear Medicine and Molecular Imaging - Physics}.

\item  Management Committee Member to \href{http://www.cost.eu/COST_Actions/mpns/TD1007}{EU COST Action TD1007} for "Bimodal PET-MRI molecular imaging technologies and applications for in vivo monitoring of disease and biological processes".

\item Grant reviewer for the Instituto de Salud Carlos III (Spain), in the modality of Technological Developments for Health Applications (AES 2015).

\end{description}

\bloque{Skills}
\begin{itemize}

\item Imaging techniques: (TOF)-PET/CT, PET/MR, and SPECT.

\item Image processing software: FreeSurfer, SPM, PMOD, Amide, and FSL.
 
\item Experience in international, large-scale R\&D environments. 

\item Experience working in multi-disciplinary teams with members from academia, industry, and clinics.

\item Principal Investigator in 3 research grants and collaborator in other 11 grants and contracts with industry.

\item Authored 60+ scientific publications and delivered 20+ invited talks and conference presentations.

\item Radioactive Facilities Supervisor License, Spanish Nuclear Safety Council, 2009.  

\item Languages: Native Spanish, native Galician, native competence in English, and basic German.

\end{itemize}


\bloque{Honors \& Awards}
\begin{description}

\item[] M+Vision Advanced Fellowship (declined), Madrid-MIT M+Vision Consortium  \hfill \textit{Aug. 2013}

\item[] M+Vision Fellowship, Madrid-MIT M+Vision Consortium   \hfill \textit{Jul. 2012}\\
    Ranked 2nd among more than 100 applicants.
    
\item[] ``Best Oral Presentation in Nuclear Medicine'', Spanish Nuclear Society Annual Meeting \hfill\textit{Oct. 2010}
     
\item[] Front cover of the Journal of Nuclear Medicine, Vol. 50 \hfill \textit{Aug. 2009} \\
    Featuring results on ``Impact of Time-of-Flight on PET Tumor Detection''.

\item[] Jos\'e Castillejo Fellowship, Ministerio de Educaci\'on y Ciencia, Spain \hfill \textit{Sep. 2007--Jun. 2008}

\item[] \'Angeles Alvari\~no Fellowship, Xunta de Galicia, Spain \hfill \textit{Dec. 2008--Dec. 2011}\\
    Ranked 2nd among more than 200 applicants.

\item[] Predoctoral research grants\\
    Universidad de Santiago de Compostela, Spain \hfill  \textit{ Oct. 2005--May 2006}\\
    Excma. Diputaci\'on Provincial de A Coru\~na, Spain \hfill \textit{Jul. 2003--Jul. 2004}\\
    Excma. Diputaci\'on Provincial de A Coru\~na, Spain \hfill \textit{Jul. 2002--Jul. 2003}

\end{description}

\bloque{Selected Publications}
\textit{See also my Google Scholar profile at}
\href{http://bit.ly/cloispubs}{\texttt{bit.ly/cloispubs}}.

\begin{itemize} 

 \item  F. D. Popota, P. Aguiar, S. Espa\~na, \textbf{C. Lois}, J. M. Udias, D. Ros, J. Pavia, J. D. Gispert, "Monte Carlo simulations versus experimental measurements in a small animal PET system. A comparison in the NEMA NU 4-2008 framework ", Physics in Medicine and Biology 60 (1), 151 (2015). 
   
 \item P. Aguiar, J. Silva-Rodriguez, D. M. Gonzalez-Castano, F. Pino, M. Sanchez, M. Herranz, A. Iglesias, \textbf{C. Lois}, A. Ruibal, ``A portable device for small animal SPECT imaging in clinical gamma-cameras'', \textit{JINST} \textbf{9} P07004 (2014).

\item \textbf{C. Lois}, I. Bezrukov, H. Schmidt, N. Schwenzer, M.~K. Werner, J. Kupferschl�ger, T. Beyer ``Effect of MR contrast agents on quantitative
        accuracy of PET in combined whole-body PET/MR imaging'', \textit{Eur. J.
        Nucl. Med.}, \textbf{39}, 1756 (2012).

\item P. Aguiar and \textbf{C. Lois}, ``Analytical study of the effect of the system geometry on photon sensitivity and depth of interaction of positron emission mammography", Journal of Oncology (2012).

\item \textbf{C. Lois}, B.~W. Jakoby, M.~J. Long, K.~F. Hubner, D.~W. Barker, M.~E. Casey, M. Conti, V.~Y. Panin, D.~J. Kadrmas, D.~W Townsend , ``An assessment of the impact of incorporating
        Time-of-Flight (TOF) into clinical PET/CT imaging'', \textit{J. Nucl.
        Med.}, \textbf{51}, 1315 (2010). 
    
\item  D.~J. Kadrmas, M.~E. Casey, M. Conti, B.~W. Jakoby, \textbf{C. Lois}, D.~W. Townsend  , ``Impact of Time-of-Flight on PET Tumor
        Detection'', \textit{J. Nucl. Med.}, \textbf{50},  3104 (2009).
  
\item AA. Alves {\em et al.} (The LHCb Collaboration), "The LHCb Detector at the LHC", Journal of Instrumentation 3 (2008).
\end{itemize}


Authored 60+ scientific publications, 10+ as first or last author, and with a total of 3200+ citations.



\end{document}
